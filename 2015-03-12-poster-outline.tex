\documentclass[11pt]{article}

\usepackage{url,amssymb,epsfig,color,xspace,enumerate}
\usepackage[a4paper,margin=1.5cm,footskip=.5cm]{geometry}
\usepackage{amsmath}
\usepackage{mathtools}
\DeclarePairedDelimiter\ceil{\lceil}{\rceil}
\DeclarePairedDelimiter\floor{\lfloor}{\rfloor}
\usepackage[pdftitle={CREA Experimental Biology Boston Poster Outline},%
  pdfsubject={Knowledge Extraction from PubMed Using Natural Language Processing},%
  pdfauthor={Teyden Nguyen}]{hyperref}
  
\newcommand{\tab}{\hspace*{10 mm}}

\begin{document}

\section{Notes:}
\begin{itemize}
\item Each section must be 9-12 sentences
\item Text orientation must be planned when writing is complete
\item Add images or graphs as examples
\item If you wish to write a particular section, make a comment indicating which point + section below
\item Cite where you get external information from (critical)
\end{itemize}

\begin{center}
  \vspace{3mm}
         {\bf Using natural language parsing and artificial intelligence techniques to initiate a phase change to biological knowledge}\\
\end{center}

\section{Abstract}

The Biological Navigation and Visualization Tool (BioNaVisT) is an artificial intelligence system that can read, interpret, and organize information from textbooks and research papers. The project is currently being developed to tackle the rapid growth of biological discoveries and literature. Today as individuals, it is becoming infeasible to keep up with the amount of information that is published at an ever increasing rate. The BioNaVisT that is being developed may allow us to interact with very large sets of data. The system draws together vast amounts of information disseminated by researchers, and creates a visual user-interface to show concentrations of related data using a natural language processing program and a customized open-source graphing program. The system will utilize artificial intelligence to establish connections between different blocks of information, and allow the researcher to see clusters of information and display the relative importance of each node in the system. This system could help shed light on very complex biological process such as the human aging process.

\section{Introduction}

\begin{itemize}
\item Motive: No complete model on human aging. Several different theories, lots of controversy.
\item Problems:
	\begin{itemize}
		\item \textbf{Difficulty in Quantifying Aging:} Complex subject, hard to investigate experimentally. Too many systems involved, dynamic biological interactions, too much variation between individuals, too much variation over time.
		\item \textbf{"Big Data":} Large number of abstracts on Pubmed. New articles on aging published daily. Data stored in inaccessible formats, as semantically unstructured text or images. 
	\end{itemize}
\item Proposed Solution: 
	\begin{itemize}
		\item \textbf{Efficiency:} Automate text curation process
		\item \textbf{Reliability:} Extract knowlege on aging processes and improve results with machine learning
		\item \textbf{Functionality:} Provide passage for concept modelling and hypothesis generation
	\end{itemize}
\end{itemize}

\section{Background}

\begin{itemize}
\item \textbf{Quantifying Aging Processes:} Quantify interactions in human body (growth in biological aging, is a function of time). Interested in biological events to build our model. CREA is modelling aging from a systems biology standpoint. Begin by looking at interactions between biological objects, how they give rise to a model for function and behaviour of a particular system, and how the sum of their parts give rise to function and behavior of all systems as a whole.  
\item \textbf{Algorithmic Approach:} NLP to extract unstructured text from PubMed article abstracts. Store relations in structured digital format. Map data using graphs to visualize connections between unlinked data.
\item \textbf{Expected Advantages:} Improve pattern discovery. Provide knowledge base accessibility, permit data manipulation such as applying functions that predict biological change in a process. 
\end{itemize}

\section{Methods}

\begin{itemize}
\item \textbf{Knowledge Extraction and NLP}
	\begin{itemize}
	\item Use a NLP engine to process and extract relations of PubMed article abstracts
	\item Relations are a 3-tuple of (subject, predicate, object)
	\item Filter these relations
	\item Plot them on a graph
	\item Analyse the graph to find patterns, or possible optimizations to the algorithm
	\item Repeat
	\end{itemize}
\item \textbf{Current: Bootstrapping / Semi-Supervised Learning}
	\begin{itemize}
	\item Iterate: Use patterns to get more instances and patterns 
	\item (Example 1)	
	\item 		Target relation: chemical interaction
	\item 		Seed tuple: [named_entity=dopamine, named_entity=progesterone] ...
	\item (Example 2)	
	\item 		Target relation: excitation event
	\item 		Seed tuple: [named_entity=dopamine, verb=excite]	
\item \textbf{Entity Classification}
	\begin{itemize}
	\item Biological Events [not complete, made this up on the spot]
		\begin{itemize}
		\item Pathways (i.e. Wnt signaling)
		\item Biomolecular Interaction (i.e. glucose increase insulin release)
		\item Biological Phenomena (i.e. Apoptosis - cell death)
		\item Environmental Events (i.e. Acetaminophen digestion)
		\item Others* (i.e. Binding in cytosol)
		\end{itemize}
	\end{itemize}
\end{itemize}

\section{Results}

\begin{itemize}
	\item To be discussed Saturday
\end{itemize}

\section{Discussion}

\textbf{NLP:} Effectiveness of using computer science data structures (particularly graph structures) to store knowledge

\begin{itemize}
\item Explore connections of known relations to find new ones mentioned in new papers
\item Find papers that back up a relation link
\item Find papers that support or contradict each other edge
\item Find the importance of something, i.e. its multitude of relations to others histogram
\end{itemize}

\textbf{Web App:} Supporting transparency of our research

\begin{itemize}
\item Visualizing the relations to easily identify connections graph
\item A tool to find large-scale trends and relations previously overlooked
\item An simpler interface for anyone to use interaction
\item A way to test and reflect the well-roundedness of this method\end{itemize}
\end{itemize}

\section{Conclusion}
\begin{itemize}
\item Describe what we learned about aging as a result of the methods we used.
\item How the methods we used help us to accomplish our task, and the possibility of generalizing it to help apply it to related studies.
\item Describe how we can improve the process.
\end{itemize}

\section{Bibliography}



\section{Appendices}
