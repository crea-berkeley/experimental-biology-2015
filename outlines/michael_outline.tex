\documentclass[11pt]{article}

\usepackage{url,amssymb,epsfig,color,xspace,enumerate}
\usepackage[a4paper,margin=1.5cm,footskip=.5cm]{geometry}
\usepackage{amsmath}
\usepackage{mathtools}
\DeclarePairedDelimiter\ceil{\lceil}{\rceil}
\DeclarePairedDelimiter\floor{\lfloor}{\rfloor}
\usepackage[pdftitle={CREA Experimental Biology Boston Poster Outline},%
  pdfsubject={Knowledge Extraction from PubMed Using Natural Language Processing},%
  pdfauthor={Zhucheng Tu}]{hyperref}
  
\newcommand{\tab}{\hspace*{10 mm}}

\begin{document}

\begin{center}
  \vspace{3mm}
         {\bf Discovering New Knowledge on the Neuroendocrine System by Extracting Knowledge from PubMed using Natural Language Processing}\\
\end{center}

\section{Abstract}

This section will briefly summarize:

\begin{itemize}
\item The need for using AI/NLP to gather biomedical knowledge, using neuroendocrine system as an example.
\item The methods the group used to gather, process, and visualize knowledge.
\item The impact of using these methods, and how they can be improved in the future.
\end{itemize}

\section{Introduction and Background}

\begin{itemize}
\item Discuss the need to use software to help researchers to see the progress made in knowledge about the neuroendocrine system.
\item This is essentially a ``Big Data" problem.
	\begin{itemize}
		\item \textbf{Volume:} Large number of abstracts on PubMed
		\item \textbf{Velocity:} New articles on aging published daily
		\item \textbf{Variety:} Text, illustrations, images, etc. (although we only deal with textual data) 
	\end{itemize}
\item Briefly summarize how the group approaches the problem.
\end{itemize}

\section{Knowledge Extraction and NLP}

\begin{itemize}
\item How we use keywords to filter and download abstracts, streaming in abstracts as they are published (RSS/Feedly?).
\item How we use Stanford NLP library to parse sentences.
\item How we perform pattern matching on the parse tree in Scala to derive relations.
\end{itemize}

\section{Relation Processing}
\begin{itemize}
\item How we can compile relations and export them to CSV format.
\item How we filter relations, or triples, using a limited set of predicates and subject/objects that are only cells and substances.
\item How we can relate relations to the body anatomy tree.
\end{itemize}

\section{Relation Visualization}
\begin{itemize}
\item Data representation using nodes and edges, and how clustering can be applied.
\item Visualization using Gephi.
\item Description of web app features.
\item Discuss interesting results we found through the data visualization.
\end{itemize}

\section{Conclusion}
\begin{itemize}
\item Describe what we learned about knowledge of neuroendocrine system as a result of the methods we used.
\item How the methods we used help us to accomplish our task, and the possibility of generalizing it to help apply it to related studies.
\item Describe how we can improve the process.
\end{itemize}

\section{Bibliography}

\section{Appendices}

\end{document}